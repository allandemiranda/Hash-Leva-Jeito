Desenvolvido por \href{https://github.com/allandemiranda}{\tt Allan de Miranda}

25 de setembro de 2019 -\/ Segurança de Redes -\/ I\+MD

\subsection*{Introdução}

Por conta de seus conhecimentos em segurança computacional, você foi convocado a colaborar com as investigações da operação “\+Leva Jeito” na qualidade de concultor técnico.

Sua contribuição inicial será o desenvolvimento de um software que permita garantir a integridade de documentos extraídos de computadores, preservando assim a cadeia de custódia destes elementos de prova.

O problema relatado é que documentos extraídos ou coletados em computadores de suspeitos podem potencialmente serem alterados entre a sua coleta e o seu processamento investigativo.

Assim, a equipe da “\+Leva Jeito” requisitou a você o desenvolvimento de um software que permita verificar a integridade dos dados a qualquer momento.

\subsection*{Implementação}

Implemente o programa guarda que, usando de calculo de puro Hash ou H\+M\+AC, permita garantir a autenticação de um conjunto de arquivos para uma determinada pasta (recursivamente). O programa deverá ser executado em linha de comando, seguindo a sintaxe\+:

./guarda $<$metodo$>$ $<$opcao$>$ $<$pasta$>$ $<$saida$>$

― $<$metodo$>$ \+: indica o método a ser utilizado ( --hash ou --hmac senha) ― $<$opcao$>$\+: indica a ação a ser desempenhada pelo programa • -\/i \+: inicia a guarda da pasta indicada em $<$pasta$>$, ou seja, faz a leitura de todos os arquivos da pasta (recursivamente) registrando os dados e Hash/\+H\+M\+AC de cada um e armazenando numa estrutura própria (Ex\+: tabela hash em uma subpasta oculta ./guarda – ou pode ser usada uma árvore B) • -\/t \+: faz o rastreio (tracking) da pasta indicada em $<$pasta$>$, inserindo informações sobre novos arquivos e indicando alterações detectadas/exclusões • -\/x \+: desativa a guarda e remove a estrutura alocada ― $<$pasta$>$ \+: indica a pasta a ser “guardada” ― $<$saida$>$ \+: indica o arquivo de saída para o relatório (-\/o saída). Caso não seja passado este parâmetro, a saída deve ser feita em tela.

\subsection*{Requisitos}

Para a instalação do programa é necessário que seu computador tenha os seguintes pacotes\+:

{\ttfamily g++ openssl libssl-\/dev libboost-\/all-\/dev}

Para instalar estes pacotes abra o terminal e digite\+:

{\ttfamily sudo apt-\/get install g++ openssl libssl-\/dev libboost-\/all-\/dev}

\subsection*{Instalação}

Para instalar e executar o programa siga as instruções\+:


\begin{DoxyEnumerate}
\item Abra o terminal do seu sistema operacional e digite {\ttfamily cd}
\item {\ttfamily git clone \href{https://github.com/allandemiranda/Hash-Leva-Jeito.git}{\tt https\+://github.\+com/allandemiranda/\+Hash-\/\+Leva-\/\+Jeito.\+git}}
\item {\ttfamily cd Hash-\/\+Leva-\/\+Jeito/}
\item {\ttfamily make}
\end{DoxyEnumerate}

\subsection*{Como utilizar programa \hyperlink{class_guarda}{Guarda}}


\begin{DoxyEnumerate}
\item Abra o terminal do seu sistema operacional e digite {\ttfamily cd}
\item {\ttfamily cd criptografia-\/simetrica/}
\item {\ttfamily ./bin/guarda $<$metodo$>$ $<$opcao$>$ $<$pasta$>$ $<$saída$>$}
\item A utilização destes argumentos de entrada estão descritos no item de implementação. 
\end{DoxyEnumerate}